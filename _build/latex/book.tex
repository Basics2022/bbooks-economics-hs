%% Generated by Sphinx.
\def\sphinxdocclass{jupyterBook}
\documentclass[letterpaper,10pt,italian]{jupyterBook}
\ifdefined\pdfpxdimen
   \let\sphinxpxdimen\pdfpxdimen\else\newdimen\sphinxpxdimen
\fi \sphinxpxdimen=.75bp\relax
\ifdefined\pdfimageresolution
    \pdfimageresolution= \numexpr \dimexpr1in\relax/\sphinxpxdimen\relax
\fi
%% let collapsible pdf bookmarks panel have high depth per default
\PassOptionsToPackage{bookmarksdepth=5}{hyperref}
%% turn off hyperref patch of \index as sphinx.xdy xindy module takes care of
%% suitable \hyperpage mark-up, working around hyperref-xindy incompatibility
\PassOptionsToPackage{hyperindex=false}{hyperref}
%% memoir class requires extra handling
\makeatletter\@ifclassloaded{memoir}
{\ifdefined\memhyperindexfalse\memhyperindexfalse\fi}{}\makeatother

\PassOptionsToPackage{warn}{textcomp}

\catcode`^^^^00a0\active\protected\def^^^^00a0{\leavevmode\nobreak\ }
\usepackage{cmap}
\usepackage{fontspec}
\defaultfontfeatures[\rmfamily,\sffamily,\ttfamily]{}
\usepackage{amsmath,amssymb,amstext}
\usepackage{polyglossia}
\setmainlanguage{italian}



\setmainfont{FreeSerif}[
  Extension      = .otf,
  UprightFont    = *,
  ItalicFont     = *Italic,
  BoldFont       = *Bold,
  BoldItalicFont = *BoldItalic
]
\setsansfont{FreeSans}[
  Extension      = .otf,
  UprightFont    = *,
  ItalicFont     = *Oblique,
  BoldFont       = *Bold,
  BoldItalicFont = *BoldOblique,
]
\setmonofont{FreeMono}[
  Extension      = .otf,
  UprightFont    = *,
  ItalicFont     = *Oblique,
  BoldFont       = *Bold,
  BoldItalicFont = *BoldOblique,
]



\usepackage[Sonny]{fncychap}
\ChNameVar{\Large\normalfont\sffamily}
\ChTitleVar{\Large\normalfont\sffamily}
\usepackage[,numfigreset=1,mathnumfig]{sphinx}

\fvset{fontsize=\small}
\usepackage{geometry}


% Include hyperref last.
\usepackage{hyperref}
% Fix anchor placement for figures with captions.
\usepackage{hypcap}% it must be loaded after hyperref.
% Set up styles of URL: it should be placed after hyperref.
\urlstyle{same}

\addto\captionsitalian{\renewcommand{\contentsname}{Microeconomia}}

\usepackage{sphinxmessages}



        % Start of preamble defined in sphinx-jupyterbook-latex %
         \usepackage[Latin,Greek]{ucharclasses}
        \usepackage{unicode-math}
        % fixing title of the toc
        \addto\captionsenglish{\renewcommand{\contentsname}{Contents}}
        \hypersetup{
            pdfencoding=auto,
            psdextra
        }
        % End of preamble defined in sphinx-jupyterbook-latex %
        

\title{Economia per le scuole superiori}
\date{27 dic 2024}
\release{}
\author{basics}
\newcommand{\sphinxlogo}{\vbox{}}
\renewcommand{\releasename}{}
\makeindex
\begin{document}

\pagestyle{empty}
\sphinxmaketitle
\pagestyle{plain}
\sphinxtableofcontents
\pagestyle{normal}
\phantomsection\label{\detokenize{intro::doc}}


\sphinxAtStartPar
Questo libro fa parte del materiale pensato per \sphinxhref{https://basics2022.github.io/bbooks-hs}{le scuole superiori}





\sphinxstepscope


\part{Microeconomia}

\sphinxstepscope


\chapter{Introduzione alla microeconomia}
\label{\detokenize{ch/micro:introduzione-alla-microeconomia}}\label{\detokenize{ch/micro:economics-hs-micro}}\label{\detokenize{ch/micro::doc}}
\sphinxstepscope


\part{Macroeconomia}

\sphinxstepscope


\chapter{Introduzione alla macroeconomia}
\label{\detokenize{ch/macro:introduzione-alla-macroeconomia}}\label{\detokenize{ch/macro:economics-hs-macro}}\label{\detokenize{ch/macro::doc}}

\section{Introduzione}
\label{\detokenize{ch/macro:introduzione}}\label{\detokenize{ch/macro:economics-hs-macro-intro}}\subsubsection*{Soggetti economici:}

\sphinxAtStartPar
individui, aziende, settore finanziario, governo, estero (five\sphinxhyphen{}sector \sphinxstylestrong{circular flow model})

\sphinxAtStartPar
I \sphinxhyphen{}> A: risorse (tempo, capacità) in cambio di reddito, spesa per prodotti
I <\sphinxhyphen{} A: reddito per risorse, prodotti in cambio di denaro
I \sphinxhyphen{}> F: risparmi
I \sphinxhyphen{}> G: tasse
I \sphinxhyphen{}> E: import
A <\sphinxhyphen{} F: investimenti
A <\sphinxhyphen{} G: spesa pubblica (parte, poiché parte rimane per spese dipendenti pubblici?)
A <\sphinxhyphen{} E: export

\sphinxAtStartPar
Astrazione utile per misura PIL, riassunto dipendenze tra settori,…
\subsubsection*{Misura dell’economia}
\begin{itemize}
\item {} 
\sphinxAtStartPar
In termini di cosa si misura l’economia? In beni e servizi (no a illusioni monetaristiche)

\end{itemize}
\subsubsection*{Moneta}
\begin{itemize}
\item {} 
\sphinxAtStartPar
Con cosa si misura? Con un valuta (no a illusioni monetaristiche, caratteristiche valuta \sphinxhyphen{} il BTC ha queste caratteristiche?,…)

\end{itemize}
\subsubsection*{…}
\subsubsection*{Storia del pensiero e delle teorie economiche}
\begin{itemize}
\item {} 
\sphinxAtStartPar
…

\item {} 
\sphinxAtStartPar
classici del XVIII e XIX secolo:
\begin{itemize}
\item {} 
\sphinxAtStartPar
A.Smith

\item {} 
\sphinxAtStartPar
Positivismo e utilitarismo: J.Bentham, D.Ricardo, J.Stuart Mill

\item {} 
\sphinxAtStartPar
…

\item {} 
\sphinxAtStartPar
LSE

\end{itemize}

\item {} 
\sphinxAtStartPar
neoclassici:
\begin{itemize}
\item {} 
\sphinxAtStartPar
anglo\sphinxhyphen{}american

\item {} 
\sphinxAtStartPar
…

\item {} 
\sphinxAtStartPar
scuola di Vienna: von Hayek,…

\item {} 
\sphinxAtStartPar
…

\end{itemize}

\item {} 
\sphinxAtStartPar
XX secolo:
\begin{itemize}
\item {} 
\sphinxAtStartPar
Keynes

\item {} 
\sphinxAtStartPar
Chicago, M.Friedman

\end{itemize}

\end{itemize}


\subsection{Misurazione dell’attività economica, PIL e PNL}
\label{\detokenize{ch/macro:misurazione-dell-attivita-economica-pil-e-pnl}}\label{\detokenize{ch/macro:economics-hs-macro-intro-econ-measure-gdp}}
\sphinxAtStartPar
Esistono 3 possibili definizioni equivalenti con 3 approcci differenti del PIL/PNL, come misura dell’attività economica in un determinato intervallo di tempo in una determinata zona:
\begin{enumerate}
\sphinxsetlistlabels{\arabic}{enumi}{enumii}{}{.}%
\item {} 
\sphinxAtStartPar
produzione: valore aggiunto di tutto l’output (beni più servizi) prodotto

\item {} 
\sphinxAtStartPar
reddito: reddito ottenuto dai produttori dell’output

\item {} 
\sphinxAtStartPar
spesa: spesa totale dei compratori dell’output

\end{enumerate}

\sphinxAtStartPar
L’equivalenza dei 3 approcci viene rappresentata dall”\sphinxstylestrong{identità fondamentale della contabilità del reddito nazionale}
\begin{equation*}
\begin{split} \text{valore aggiunto totale} = \text{ricavo totale} = \text{spesa totale} \end{split}
\end{equation*}
\sphinxAtStartPar
\sphinxstylestrong{todo} Differenze? Difficoltà di misura secondo i diversi approcci? Cosa non viene misurato


\subsection{PIL, PNL e NFP}
\label{\detokenize{ch/macro:pil-pnl-e-nfp}}\label{\detokenize{ch/macro:economics-hs-macro-intro-econ-measure-gdp-gnp}}
\sphinxAtStartPar
Il PIL (GDP) misura l’attività economica di tutti gli attori che hanno sede legale e fiscale entro i confini nazionali; il PNL(GNL) misura l’attività economica che produce output entro i confini nazionali. La differenza viene definita net factor payments from abroad, NFP, o net foreign factor income, NFFI,
\begin{equation*}
\begin{split}\text{NFP} := \text{GDP} - \text{GNP}\end{split}
\end{equation*}
\sphinxAtStartPar
Alcuni fattori che rendono GDP diverso da GNP sono aziende con produzione all’estero o lavoratori emigrati o immigrati che trasferiscono parte del reddito dentro o fuori i confini (rimesse)


\subsection{Composizione del PIL}
\label{\detokenize{ch/macro:composizione-del-pil}}\label{\detokenize{ch/macro:economics-hs-macro-intro-gdp}}\begin{equation*}
\begin{split}Y = C + I + G + NX\end{split}
\end{equation*}
\sphinxAtStartPar
Spese totali che formano il PIL, \(Y\), sono di 4 tipi:
\begin{itemize}
\item {} 
\sphinxAtStartPar
\(C\) consumi

\item {} 
\sphinxAtStartPar
\(I\) investimenti

\item {} 
\sphinxAtStartPar
\(G\) spesa governativa/pubblica

\item {} 
\sphinxAtStartPar
\(NX\) esportazione netta di beni e servizi all’estero

\end{itemize}

\begin{sphinxadmonition}{note}{Nota:}
\sphinxAtStartPar
Com’è composto il PIL nei diversi stati? Come si è evoluto negli anni?
\end{sphinxadmonition}


\subsection{Altre variabili macroeconomiche: PIL, tasso di disoccupazione, tasso di inflazione; legge di Okun e curva di Phillips}
\label{\detokenize{ch/macro:altre-variabili-macroeconomiche-pil-tasso-di-disoccupazione-tasso-di-inflazione-legge-di-okun-e-curva-di-phillips}}\label{\detokenize{ch/macro:economics-hs-macro-intro-macro-vars}}

\subsubsection{Inflazione}
\label{\detokenize{ch/macro:inflazione}}\label{\detokenize{ch/macro:economics-hs-macro-intro-macro-vars-inflation}}
\sphinxAtStartPar
\sphinxstylestrong{Price level.} Il livello dei prezzi traccia l’andamento di prezzo di un bene o servizio (o beni e servizi equivalenti) nel tempo. L’Inflazione è definita come l’aumento percentuale del livello dei prezzi. Esistono principalmente due misure del livello dei prezzi: il deflatore del PIL e il CPI. Non è detto che i due indici corrispondano: poiché il deflatore del PIL traccia l’andamento dei prezzi dei beni prodotti nell’economia, mentre il CPI traccia l’andamento dei prezzi dei beni acquistati dai consumatori in un’economia.

\sphinxAtStartPar
\sphinxstylestrong{Deflatore del PIL.} Nel periodo \(t\), il deflatore del PIL è definito come il rapporto tra il PIL nominale e il PIL reale,
\begin{equation*}
\begin{split}P_t = \frac{\$ Y_t}{Y_t} \ .\end{split}
\end{equation*}
\sphinxAtStartPar
\sphinxstylestrong{CPI, Consumer Price Index.}


\subsubsection{Legge di Okun}
\label{\detokenize{ch/macro:legge-di-okun}}\label{\detokenize{ch/macro:economics-hs-macro-intro-macro-vars-okun}}
\sphinxAtStartPar
Relazione tra variazione percentuale dell’output e variazione percentuale della disoccupazione,
\begin{equation*}
\begin{split}\frac{\bar{Y} - Y}{\bar{Y}} = c ( u - \bar{u} ) \ ,\end{split}
\end{equation*}
\sphinxAtStartPar
con:
\begin{itemize}
\item {} 
\sphinxAtStartPar
\(\bar{Y}\) output potenziale

\item {} 
\sphinxAtStartPar
\(Y\) output misurato

\item {} 
\sphinxAtStartPar
\(c\) costante di proporzionalità positiva.

\item {} 
\sphinxAtStartPar
\(u\) tasso di disoccupazione

\item {} 
\sphinxAtStartPar
\(\bar{u}\) {\hyperref[\detokenize{ch/macro:economics-hs-macro-medium-run-unemployment-phillips}]{\sphinxcrossref{\DUrole{std,std-ref}{tasso di disoccupazione naturale}}}}

\end{itemize}
\begin{equation*}
\begin{split}\frac{\Delta Y}{Y} = k - c \Delta u \ .\end{split}
\end{equation*}
\sphinxAtStartPar
Valori plausibili della retta di regressione sono \(k \simeq 0.03\), \(c \simeq -0.4\) (per quale economia in quale istante? US oggi?)


\subsubsection{Curva di Phillips}
\label{\detokenize{ch/macro:curva-di-phillips}}\label{\detokenize{ch/macro:economics-hs-macro-intro-macro-vars-phillips}}
\sphinxAtStartPar
Vedi anche curva di Phillips nell’ambito del {\hyperref[\detokenize{ch/macro:economics-hs-macro-medium-run-unemployment-phillips}]{\sphinxcrossref{\DUrole{std,std-ref}{mercato del lavoro nel medio periodo}}}}.


\section{Il breve periodo}
\label{\detokenize{ch/macro:il-breve-periodo}}\label{\detokenize{ch/macro:economics-hs-macro-short-run}}

\subsection{Il mercato dei beni}
\label{\detokenize{ch/macro:il-mercato-dei-beni}}\label{\detokenize{ch/macro:economics-hs-macro-short-run-goods-market}}

\subsubsection{Produzione = Domanda}
\label{\detokenize{ch/macro:produzione-domanda}}\label{\detokenize{ch/macro:economics-hs-macro-short-run-goods-market-y-z}}
\sphinxAtStartPar
Domanda di beni \(Z\), offerta di beni \(Y\)
\begin{equation*}
\begin{split}Z = C + I + G + \text{NX} - \text{In}\end{split}
\end{equation*}\begin{itemize}
\item {} 
\sphinxAtStartPar
Consumo \(C\).
\begin{itemize}
\item {} 
\sphinxAtStartPar
Dipende dal reddito disponibile \(Y_d\), cioè del reddito al netto delle tasse \(T\), \(Y_d = Y - T\),
\begin{equation*}
\begin{split}C(Y_d)\end{split}
\end{equation*}
\item {} 
\sphinxAtStartPar
\(\partial_{Y_d} C > 0\)

\end{itemize}

\item {} 
\sphinxAtStartPar
Investimenti \(I\), esclude inventario \(\text{In}\) trattato a parte

\item {} 
\sphinxAtStartPar
Spesa pubblica \(G\), esclude trasferimenti poiché non sono acquisto di beni/servizi (ma fanno parte del reddito di altri attori economici)

\item {} 
\sphinxAtStartPar
Export netto \(\text{NX} = \text{Ex} - \text{Im}\)

\item {} 
\sphinxAtStartPar
Inventario \(\text{In}\)

\end{itemize}

\sphinxAtStartPar
\sphinxstylestrong{Equilibrio.} All’equilibrio di domanda e offerta di beni, \(Y^* = Z(Y^*)\),
\begin{equation*}
\begin{split}Y^* = C(Y-T) + I(Y^*,i) + G + \text{NX}\end{split}
\end{equation*}
\sphinxAtStartPar
\sphinxstylestrong{Approssimazioni.} Nel caso di funzione di consumo lineare con il reddito disponibili, \(C = c_0 + c_1 (Y- T)\),
\begin{equation*}
\begin{split}Z = c_0 + c_1 (Y-T) + I + G = \underbrace{c_0 + I + G - c_1 T}_{\text{autonomous spending}} + c_1 Y \ ,\end{split}
\end{equation*}
\sphinxAtStartPar
si può ricavare in maniera esplicita l’output in funzione della tassazione, degli investimenti e della spesa pubblica, in condizioni di equilibrio \(Y = Z\)
\begin{equation*}
\begin{split}Y^* = \frac{1}{1-c_1} \left[ c_0 + I + G - c_1 T \right] \ ,\end{split}
\end{equation*}
\sphinxAtStartPar
e valutare le sensibilità di \(Y\) a questi parametri/variabili. \sphinxstylestrong{todo} \sphinxstyleemphasis{bla bla sui moltiplicatori}


\subsubsection{Investimento = Risparmio}
\label{\detokenize{ch/macro:investimento-risparmio}}\label{\detokenize{ch/macro:economics-hs-macro-short-run-goods-market-i-s}}\begin{itemize}
\item {} 
\sphinxAtStartPar
Risparmio privato, \(S = Y_d - C = Y - T - C\)

\item {} 
\sphinxAtStartPar
Risparmio pubblico, \(S_{pub} = T - G\)

\end{itemize}

\sphinxAtStartPar
al netto dell’import/export e dell’inventario,
\begin{equation*}
\begin{split}\begin{aligned}
  Y & = C + I + G \\
  I & = Y - C - G = \\
    & = Y - C -  T + ( T - G ) = \\
    & = S + S_{pub}
\end{aligned}\end{split}
\end{equation*}

\subsection{I mercati finanziari}
\label{\detokenize{ch/macro:i-mercati-finanziari}}\label{\detokenize{ch/macro:economics-hs-macro-short-run-financial-market}}\begin{equation*}
\begin{split}\frac{M}{P} = Y L(i)\end{split}
\end{equation*}

\subsection{Il modello IS\sphinxhyphen{}LM}
\label{\detokenize{ch/macro:il-modello-is-lm}}\label{\detokenize{ch/macro:economics-hs-macro-short-run-is-lm}}\begin{itemize}
\item {} 
\sphinxAtStartPar
Modello IS\sphinxhyphen{}LM in economoia chiusa e aperta

\item {} 
\sphinxAtStartPar
Politiche economiche e fiscali
\begin{itemize}
\item {} 
\sphinxAtStartPar
Trappola della liquidità

\end{itemize}

\end{itemize}


\section{Il medio periodo}
\label{\detokenize{ch/macro:il-medio-periodo}}\label{\detokenize{ch/macro:economics-hs-macro-medium-run}}

\subsection{Il mercato del lavoro}
\label{\detokenize{ch/macro:il-mercato-del-lavoro}}\label{\detokenize{ch/macro:economics-hs-macro-medium-run-jobs-market}}\phantomsection\label{\detokenize{ch/macro:economics-hs-macro-medium-run-jobs-market-stats}}\subsubsection*{Misure nel mercato del lavoro}
\begin{itemize}
\item {} 
\sphinxAtStartPar
Forza lavoro \(L\), occupazione \(N\), disoccupazione, \(U\), tasso di disoccupazione \(u\); ore lavorate;…

\end{itemize}
\begin{equation*}
\begin{split}L = N + U\end{split}
\end{equation*}\begin{equation*}
\begin{split}u := \frac{U}{L} = 1 - \frac{N}{L}\end{split}
\end{equation*}\begin{equation*}
\begin{split}N = L (1 - u)\end{split}
\end{equation*}\phantomsection\label{\detokenize{ch/macro:economics-hs-macro-medium-run-jobs-market-wages}}\subsubsection*{Determinazione dei salari}
\begin{equation}\label{equation:ch/macro:eq:wages-prices}
\begin{split}W = P^e F(u,z)\end{split}
\end{equation}\begin{itemize}
\item {} 
\sphinxAtStartPar
\(W\) livello nominale dei salari aggregati

\item {} 
\sphinxAtStartPar
\(P^e\) livelli di prezzo attesi (vedi {\hyperref[\detokenize{ch/macro:economics-hs-macro-extra-expectations}]{\sphinxcrossref{\DUrole{std,std-ref}{aspettative \sphinxhyphen{} prezzi}}}}), poiché il lavoratore è interessato (o dovrebbe esserlo) alla retribuzione reale e non nominale

\item {} 
\sphinxAtStartPar
\(u\) tasso di disoccupazione

\item {} 
\sphinxAtStartPar
\(z\) variabile «catchall» che include tutti gli altri fattori che possono influenzare i salari; ad esempio protezione sociale per disoccupazione, inoccupazione

\end{itemize}

\sphinxAtStartPar
con:
\begin{itemize}
\item {} 
\sphinxAtStartPar
\(\partial_u F < 0\), all’aumentare della disoccupazione diminuisce il potere contrattuale (aggregato) dei lavoratori

\item {} 
\sphinxAtStartPar
\(\partial_z F > 0\) per definizione

\end{itemize}
\subsubsection*{Determinazione dei prezzi}
\begin{equation*}
\begin{split}Y = A N\end{split}
\end{equation*}\begin{itemize}
\item {} 
\sphinxAtStartPar
\(Y\) output, \(\left[\$\right]\)

\item {} 
\sphinxAtStartPar
\(N\) occupazione, \(\left[\text{n. ore lavorate}\right]\) o altre {\hyperref[\detokenize{ch/macro:economics-hs-macro-medium-run-jobs-market-stats}]{\sphinxcrossref{\DUrole{std,std-ref}{misure del lavoro}}}}

\item {} 
\sphinxAtStartPar
\(A\) produttività, \(\left[\frac{\$}{\text{n. ore lavorate}}\right]\), o riferito a altre misure del lavoro; dipende dallo {\hyperref[\detokenize{ch/macro:economics-hs-macro-long-run-progress}]{\sphinxcrossref{\DUrole{std,std-ref}{sviluppo tecnologico}}}} e influenza la crescita economica

\end{itemize}

\sphinxAtStartPar
Prezzi, in funzione del livello dei salari, del numero di dipendenti, del costo della materia prima, e del markup \(m\) dell’azienda \(\frac{\text{prezzo}}{\text{cost}}\), che dipende dal potere dell’azienda di fare il prezzo nel mercato. Nell’ipotesi che il costo della materia prima possa essere incorporato nel markup \(m\), si può (\sphinxstyleemphasis{sì? in quali condizioni? ha senso nascondere il costo della materia prima nel markup? In generale non è detto che questo sia un effetto lineare con i salari…})
\begin{equation}\label{equation:ch/macro:eq:prices-wages}
\begin{split}P = (1+m) W\end{split}
\end{equation}
\sphinxAtStartPar
Usando le due relazioni \eqref{equation:ch/macro:eq:wages-prices}, \eqref{equation:ch/macro:eq:prices-wages} si può determinare il punto di equilibrio in cui il valore del rapporto \(\left(\frac{W}{P} \right)\) è uguale nella formazione dei salari e dei prezzi,
\begin{equation}\label{equation:ch/macro:eq:wages-prices:equil}
\begin{split}\begin{aligned}
 \left( \frac{W}{P} \right)_{\text{wages}} & = \left( \frac{W}{P} \right)_{{prices}} \\
 \frac{P^e}{P} F(u, z) & = \frac{1}{1+m} \\
\end{aligned}\end{split}
\end{equation}
\sphinxAtStartPar
da cui si ricava la definizione di \sphinxstylestrong{tasso di disoccupazione naturale}, \(u_n\), come il tasso di disoccupazione per il quale è valida la condizione di equilibrio \eqref{equation:ch/macro:eq:wages-prices:equil}, a condizioni fissate di \(z\), \(\frac{P^e}{P}\) (legata all’inflazione attesa), \(m\).

\sphinxAtStartPar
\sphinxstylestrong{Influenza di \(z\), \(\pi\), \(m\) sui salari reali \(\frac{W}{P}\) e su tasso di disoccupazione naturale.}
…\sphinxstylestrong{todo}…


\subsection{Offerta e domanda aggregata: il modello AS\sphinxhyphen{}AD}
\label{\detokenize{ch/macro:offerta-e-domanda-aggregata-il-modello-as-ad}}\label{\detokenize{ch/macro:economics-hs-macro-medium-run-as-ad}}
\sphinxAtStartPar
\sphinxstylestrong{AS, Aggregate Supply.} Dalle relazioni per la formazione dei salari \eqref{equation:ch/macro:eq:wages-prices} e dei prezzi \eqref{equation:ch/macro:eq:prices-wages}, eliminando la variabile dei salari, \(W\), ed usando la relazione tra tasso di disoccupazione e output
\begin{equation}\label{equation:ch/macro:eq:as}
\begin{split}P = P^e (1+m) F(u,z) = P^e (1+m) \left( 1-\frac{Y}{A L}, z \right)\end{split}
\end{equation}
\sphinxAtStartPar
Influenza dei parametri:
\begin{itemize}
\item {} 
\sphinxAtStartPar
\(\partial_{P^e} P = (1+m) F(u,z) > 0\)

\item {} 
\sphinxAtStartPar
\(\partial_{m} P = P^e F(u,z) > 0\)

\item {} 
\sphinxAtStartPar
\(\partial_{z} P = P^e (1+m) \partial_z F > 0\)

\item {} 
\sphinxAtStartPar
\(\partial_{Y} u = - \frac{1}{A L} < 0\), e quindi \(\partial_Y P = - P^e (1+m) \frac{1}{AL} \partial_u F > 0\)

\item {} 
\sphinxAtStartPar
\(\partial_{A} u = \frac{Y}{A^2 L} > 0\), e quindi \(\partial_Y P = P^e (1+m) \frac{1}{A^2 L} \partial_u F < 0\)

\end{itemize}

\sphinxAtStartPar
\sphinxstylestrong{AD, Aggregate Demand.} Dal {\hyperref[\detokenize{ch/macro:economics-hs-macro-short-run-goods-market}]{\sphinxcrossref{\DUrole{std,std-ref}{mercato dei beni}}}}, l’output dipende dal consumo (che a sua volta dipende dal reddito disponibile, \(Y- T\)), dagli investimenti (che dipendono dall’output e dai tassi di interesse) e dalla spesa pubblica, \(Y = C(Y-T) + I(Y,i) + G + NX\). Dal {\hyperref[\detokenize{ch/macro:economics-hs-macro-short-run-financial-market}]{\sphinxcrossref{\DUrole{std,std-ref}{mercato finanziario}}}}, \(\frac{M}{P} = Y L(i)\). Assumendo trascurabile il contributo dell’export netto, \sphinxstylestrong{todo}…
\begin{equation}\label{equation:ch/macro:eq:ad}
\begin{split}Y = Y\left(\frac{M}{P}, G, T \right)\end{split}
\end{equation}
\sphinxAtStartPar
Influenza dei parametri:
\begin{itemize}
\item {} 
\sphinxAtStartPar
…

\end{itemize}

\sphinxAtStartPar
\sphinxstylestrong{Equilibrio nel corto e medio periodo.}


\subsection{Il tasso naturale di disoccupazione e la curva di Phillips}
\label{\detokenize{ch/macro:il-tasso-naturale-di-disoccupazione-e-la-curva-di-phillips}}\label{\detokenize{ch/macro:economics-hs-macro-medium-run-unemployment-phillips}}
\sphinxAtStartPar
Vedi anche {\hyperref[\detokenize{ch/macro:economics-hs-macro-intro-macro-vars-phillips}]{\sphinxcrossref{\DUrole{std,std-ref}{Introduzione alla curva di Phillips}}}}

\sphinxAtStartPar
\sphinxstylestrong{Tasso naturale di disoccupazione.}


\subsection{Inflazione, produzione e crescita della moneta}
\label{\detokenize{ch/macro:inflazione-produzione-e-crescita-della-moneta}}

\section{Il lungo periodo}
\label{\detokenize{ch/macro:il-lungo-periodo}}\label{\detokenize{ch/macro:economics-hs-macro-long-run}}\begin{itemize}
\item {} 
\sphinxAtStartPar
Storia

\end{itemize}


\subsection{Risparmio, accumulazione di capitale e produzione}
\label{\detokenize{ch/macro:risparmio-accumulazione-di-capitale-e-produzione}}\label{\detokenize{ch/macro:economics-hs-macro-long-run-savings}}

\subsection{Progresso tecnologico e crescita}
\label{\detokenize{ch/macro:progresso-tecnologico-e-crescita}}\label{\detokenize{ch/macro:economics-hs-macro-long-run-progress}}
\sphinxAtStartPar
L’output \(Y\) risulta funzione dei capitali \(K\) e del lavoro (nell’effetto combinato di occupazione, \(N\), e produttività, \(A\))
\begin{equation}\label{equation:ch/macro:eq:progress-output}
\begin{split}Y = Y(K,N,A) = Y(K, NA)\end{split}
\end{equation}
\sphinxAtStartPar
Il progresso tecnologico fa aumentare \(A\). Si può riscrivere la relazione \eqref{equation:ch/macro:eq:progress-output} in funzione di output e capitali per \sphinxstyleemphasis{lavoratore «effettivo»}, \(AN\),
\begin{equation*}
\begin{split}\frac{Y}{AN} = y \left( \frac{K}{NA} \right)\end{split}
\end{equation*}
\sphinxAtStartPar
Se gli investimenti uguagliano i risparmi dei privati e il tasso di risparmio è costante, allora \(I = S = s Y\). Dividendo per \(AN\) si ottiene
\begin{equation*}
\begin{split}\frac{I}{AN} = s \frac{Y}{AN}\end{split}
\end{equation*}

\section{Altro}
\label{\detokenize{ch/macro:altro}}\label{\detokenize{ch/macro:economics-hs-macro-extra}}

\subsection{Aspettative}
\label{\detokenize{ch/macro:aspettative}}\label{\detokenize{ch/macro:economics-hs-macro-extra-expectations}}\begin{itemize}
\item {} 
\sphinxAtStartPar
Tassi di interesse

\item {} 
\sphinxAtStartPar
Mercati finanziari

\item {} 
\sphinxAtStartPar
Consumo e investimento

\item {} 
\sphinxAtStartPar
Produzione e politica economica

\item {} 
\sphinxAtStartPar
{\hyperref[\detokenize{ch/macro:economics-hs-macro-medium-run-jobs-market-wages}]{\sphinxcrossref{\DUrole{std,std-ref}{Salari}}}}, il livello di prezzi attesi influenza la determinazione del livello dei salari,
\begin{equation*}
\begin{split}W = P^e F(u,z)\end{split}
\end{equation*}
\end{itemize}


\subsection{Economia aperta}
\label{\detokenize{ch/macro:economia-aperta}}\label{\detokenize{ch/macro:economics-hs-macro-extra-open}}\begin{itemize}
\item {} 
\sphinxAtStartPar
Domanda interna o estera

\item {} 
\sphinxAtStartPar
Deprezzamento, bilancia commerciale e produzione

\item {} 
\sphinxAtStartPar
Risparmio, investimento e disavanzo

\item {} 
\sphinxAtStartPar
Politica economica in economia aperta

\item {} 
\sphinxAtStartPar
Regimi di cambio: fissi o flessibili

\end{itemize}


\subsection{Patologie: crisi, elevato debito, iperinflazione}
\label{\detokenize{ch/macro:patologie-crisi-elevato-debito-iperinflazione}}\label{\detokenize{ch/macro:economics-hs-macro-extra-issues}}

\subsection{Politica economica, monetaria e fiscale}
\label{\detokenize{ch/macro:politica-economica-monetaria-e-fiscale}}\label{\detokenize{ch/macro:economics-hs-macro-extra-policy}}

\section{Basi}
\label{\detokenize{ch/macro:basi}}

\subsection{Storia: dalla necessità nel commercio e la partita doppia agli strumenti attualmente presenti}
\label{\detokenize{ch/macro:storia-dalla-necessita-nel-commercio-e-la-partita-doppia-agli-strumenti-attualmente-presenti}}

\subsection{Concetti fondamentali e definizioni}
\label{\detokenize{ch/macro:concetti-fondamentali-e-definizioni}}\begin{itemize}
\item {} 
\sphinxAtStartPar
Moneta:

\item {} 
\sphinxAtStartPar
Rimesse

\item {} 
\sphinxAtStartPar
Variazione congiunturale (rispetto al periodo precendente), tendenziale (rispetto allo stesso periodo dell’anno precedente)

\end{itemize}


\section{Attualità}
\label{\detokenize{ch/macro:attualita}}\begin{itemize}
\item {} 
\sphinxAtStartPar
Storia recente: Italia, Europa, US,…

\item {} 
\sphinxAtStartPar
Dati della situazione attuale

\end{itemize}


\section{Riferimenti}
\label{\detokenize{ch/macro:riferimenti}}\begin{itemize}
\item {} 
\sphinxAtStartPar
O.Blanchard, Macroeconomics

\item {} 
\sphinxAtStartPar
O.Blanchard, A.Amighini, F.Giavazzi, Macroeconomia \sphinxhyphen{} Una prospettiva europea

\item {} 
\sphinxAtStartPar
…

\end{itemize}

\sphinxstepscope


\part{Educazione finanziaria}

\sphinxstepscope


\chapter{Basi di educazione finanziaria}
\label{\detokenize{ch/fin-edu:basi-di-educazione-finanziaria}}\label{\detokenize{ch/fin-edu:economics-hs-fin-edu}}\label{\detokenize{ch/fin-edu::doc}}






\renewcommand{\indexname}{Indice}
\printindex
\end{document}